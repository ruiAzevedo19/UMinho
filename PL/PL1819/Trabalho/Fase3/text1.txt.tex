\documentclass{article}
\usepackage[bottom]{footmisc}

\begin{document}
Informática é um termo usado para descrever o conjunto das ciências relacionadas à coleta, 
armazenamento, transmissão e processamento de informações em meios digitais, estando incluídas 
neste grupo: a ciência da computação, os sistemas de informação, a teoria da informação, 
o processo de cálculo, a análise numérica e os métodos teóricos da representação dos 
conhecimentos e da modelagem dos problemas. Mas também a informática pode ser entendida como 
ciência que estuda o conjunto de informações e conhecimentos por meios digitais. O \underline{algoritmo}\footnote{algorithm} 
foi desenhado para que um dia uma máquina pudesse tratar informações. O uso da \underline{Internet}\footnote{Internet} também 
é um caso importante. De nada adianta pedir para um aluno fazer uma pesquisa na \underline{Internet} sem as 
devidas orientações. Cabe ao professor instruir os alunos para que estes não façam simples cópias de 
textos encontrados em sites. Apenas copiando, os alunos não vão aprender. Computador é uma máquina capaz
 de variados tipos de tratamento automático de informações ou processamento de dados. Um \underline{computador}\footnote{computer} pode 
 possuir inúmeros atributos, dentre eles armazenamento de dados, processamento de dados, cálculo em grande 
 escala, desenho industrial, tratamento de imagens gráficas, realidade virtual, entretenimento e cultura.
 Ñao
\begin{itemize}
    \item \underline{computador}
    \item \underline{Internet}
\end{itemize}
\appendix
\section{Apendice}
\begin{itemize}
\item algoritmo $\to$ Def: conjunto de operações, sequenciais, lógicas e não ambíguas, que, aplicadas a um conjunto de dados, permitem encontrar a solução para um problema num número finito de passos
\item Internet $\to$ Def: conjunto de redes de computadores que, espalhados por todas as regiões do planeta, conseguem trocar dados e mensagens utilizando um protocolo comum
\item computador $\to$ Def: aparelho eletrónico que é capaz de receber, armazenar e processar grande quantidade de informação em função de um conjunto de instruções com que é programado
\end{itemize}

\end{document}
