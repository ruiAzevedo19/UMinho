\documentclass{article}
\usepackage[utf8]{inputenc}
\usepackage[portuges]{babel}
\usepackage{listings}
\usepackage[table]{xcolor}
\usepackage{verbatim}
%%\usepackage{indentfirst}
\usepackage{hyperref}
%%\setlength{\parindent}{ex}
\usepackage{graphicx}
\usepackage{fancyhdr}
\usepackage{float}

\title{
       \vspace{60px}
       \Huge \textbf{Gestão de Vendas} \\[15px]
       \Large \textbf{Relatório do trabalho prático}
       %%\vspace{100px}
      }
\author{
        \begin{tabular}{c}
            \textbf{Grupo XXX} \\[5px]
                Joana Cruz(A76270) \\
                Etienne Costa (A76089) \\
                Rui Azevedo (A80789)
        \end{tabular}
       }
\date{Maio 2019}

\makeatletter         
\def\@maketitle{
    \begin{center}
        \huge \@title \\[4ex]
        \Large \@author \\[5ex] 
        \@date \\[8px]
        \vspace{100px}
        \includegraphics[scale=0.40]{img/uminho.png} \\
        \vspace{25px} 
        \small Sistemas Operativos \\[3px]
        Mestrado Integrado em Engenharia Informática \\[3px]
        Universidade do Minho \\
    \end{center}}
\makeatother

\begin{document}
\maketitle

\newpage
\section{Introdução}

Este projeto surge no âmbito da Unidade Curricular de Sistemas Operativos e tem como objetivo a implementação de  um sistema  de gestão de inventário e vendas.\\
O sistema é constituído por vários programas: manutenção de artigos, servidor de vendas, cliente de vendas e agregador
de dados.


\section{Proposta de Arquitetura}

Sendo um protocolo uma convenção que controla e possibilita uma conexão,comunicação e transferência de dados entre dois sistemas computacionais,dedicou-se está secção para explanar o formato das mensagens protocolares bem com as suas interações.

\subsection{Shared Memory}

O serviço de transferência rápida e fiável de dados foi implementado, a nível aplicacional, com recurso ao UDP. Sendo assim houve a necessidade de definir um datagrama UDP  que possui as seguintes características :

\iffalse
\begin{figure}[H]
    \centering
    \includegraphics[scale=0.50]{img/arquitetura.PNG}
    \caption{Arquitetura com memória partilhada}
\end{figure}
\fi


\subsection{Without Shared Memory}


\newpage
\section{Manutenção de artigos}
\subsection{Métodos}


\newpage
\section{Servidor de vendas}
Para a implementação destes conceitos, foi utilizada a linguagem de programação orientada a objectos \emph{C}, e foram definidas as seguintes classes:
\subsection{Métodos}
De modo a não tornar esta leitura exaustiva , decidiu-se simplesmente apresentar os métodos mais relevantes
e procurar explicar o impacto que cada um tem na implementação do nosso serviço.


\newpage
\section{Cliente de vendas}
\subsection{Métodos}


\newpage
\section{Agregador}
\subsection{Métodos}


\newpage
\section{Funcionalidades adicionais}





\newpage
\section{Testes e Resultados}


\newpage
\section{Conclusões e trabalho futuro}



\end{document}
